
\subsection{main functions}

\subsubsection[fc\_tools.graphics.display\_rgb function]{\fcmcode{fc_tools.graphics.display_rgb} function}
The \fcmcode{fc_tools.graphics.display_rgb} displays colors of a $n$-by-$3$ RGB colors array with their names
if available.
\paragraph{Syntaxe}
\begin{syntaxe}
fc_tools.graphics.display_rgb(rgb)
fc_tools.graphics.display_rgb(rgb, names)
\end{syntaxe}

\begin{filecontents*}{main.m}
fc_tools.graphics.display_rgb(rand(40,3))
\end{filecontents*}
\begin{filecontents*}{after.m}
fc_tools.graphics.SaveAllFigsAsFiles('display_rgb', SaveOptions{:})
\end{filecontents*}
\fcrun{figures/display_rgb}

\begin{center}
\fbox{\begin{minipage}{0.95\textwidth}
\fcimagec{figures/display_rgb_fig1_\fccmdname\fccmdversionabr}{0.7\textwidth}
\par\noparindent\FClistingresize{\lstinputlisting[caption={\fcmcode{fc_tools.graphics.display_rgb} function \label{Lst:display_rgb}}]
{figures/display_rgb_\fccmdname\fccmdversionabr.m}}
\end{minipage}}
\end{center}

\subsubsection[fc\_tools.graphics.selectColors function]{\fcmcode{fc_tools.graphics.selectColors} function}
The \fcmcode{fc_tools.graphics.selectColors} function returns colors that are maximally perceptually distinct
 without using the Image Processing Toolbox.
 
This function is inspired by the function \fcmcode{select_colors} (or \fcmcode{distinguishable_colors}) of
\textit{Timothy E. Holy} which uses the Image Processing Toolbox of Matlab.
\paragraph{Syntaxe}
\begin{syntaxe}
colors=fc_tools.graphics.selectColors(N)
colors=fc_tools.graphics.selectColors(N, key, value)
\end{syntaxe}
\paragraph{Description}
\begin{description}
\item \fbox{\fcmcode{colors=fc_tools.graphics.selectColors(N)}}
Returns \fcmcode{N} colors that are maximally perceptually distinct as a $\fcmcode{N}$-by-3 RGB colors array.
\item \fbox{\fcmcode{colors=fc_tools.graphics.selectColors(N, Name, Value)}}
specifies function options using one or more \fcmcode{Name},\fcmcode{Value} pair 
arguments. Options are
\begin{itemize}
\item[$\bullet$] \texttt{'background'} : the \fcmcode{N} colors selected will be as far as possible from the colors 
specified by this options as a $n$-by-3 RGB colors array. Default is \fcmcode{[1 1 1; 0 0 0; 0.8 0.8 0.8;1,0,1]}
\item[$\bullet$] \texttt{'func'} : To specify an other function for converting RGB colors to LAB colors.
Default is local \fcmcode{RGB2LAB} function.
\end{itemize}
\end{description}

\begin{filecontents*}{main.m}
colors1=fc_tools.graphics.selectColors(14);
figure(1)
fc_tools.graphics.display_rgb(colors1)
colors2=fc_tools.graphics.selectColors(14,'background',[1 1 1]);
figure(2)
fc_tools.graphics.display_rgb(colors2)
\end{filecontents*}
\begin{filecontents*}{after.m}
fc_tools.graphics.SaveAllFigsAsFiles('selectColors', SaveOptions{:})
\end{filecontents*}
\fcrun{figures/selectColors}

%{\tiny \verbatiminput{results/buildmesh2d01.txt}}
%\fcmatlaboutput{results/siMesh_plot01_\fccmdname\fccmdversionabr.m}{results/siMesh01.txt}
\begin{center}
\fbox{\begin{minipage}{0.95\textwidth}
\fcdblfigure{figures/selectColors_fig1_\fccmdname\fccmdversionabr}{figures/selectColors_fig2_\fccmdname\fccmdversionabr}
%\fcdblfigure{figures/plot2D_fig3_\fccmdname\fccmdversionabr}{figures/plot2D_fig4_\fccmdname\fccmdversionabr}
\par\noparindent\FClistingresize{\lstinputlisting[caption={\fcmcode{fc_tools.graphics.xcolor.selectColors} function \label{Lst:selectColors}}]
{figures/selectColors_\fccmdname\fccmdversionabr.m}}
\end{minipage}}
\end{center}

\begin{filecontents*}{main.m}
colors=fc_tools.graphics.selectColors(6);
x=0:pi/100:2*pi;
figure(1)
plot(x,cos(x),'color',colors(1,:),'Linewidth',1.5)
hold on
plot(x,sin(x),'color',colors(2,:),'Linewidth',1.5)
plot(x,cos(2*x),'color',colors(3,:),'Linewidth',1.5)
plot(x,sin(2*x),'color',colors(4,:),'Linewidth',1.5)
plot(x,cos(3*x),'color',colors(5,:),'Linewidth',1.5)
plot(x,sin(3*x),'color',colors(6,:),'Linewidth',1.5)
legend('cos(x)','sin(x)','cos(2x)','sin(2x)','cos(3x)','sin(3x)')
\end{filecontents*}
\begin{filecontents*}{after.m}
fc_tools.graphics.SaveAllFigsAsFiles('selectColors02', SaveOptions{:})
\end{filecontents*}
\fcrun{figures/selectColors02}

%{\tiny \verbatiminput{results/buildmesh2d01.txt}}
%\fcmatlaboutput{results/siMesh_plot01_\fccmdname\fccmdversionabr.m}{results/siMesh01.txt}
\begin{center}
\fbox{\begin{minipage}{0.95\textwidth}
\fcimagec{figures/selectColors02_fig1_\fccmdname\fccmdversionabr}{0.7\textwidth}
\par\noparindent\FClistingresize{\lstinputlisting[caption={\fcmcode{fc_tools.graphics.xcolor.selectColors} function \label{Lst:selectColors02}}]
{figures/selectColors02_\fccmdname\fccmdversionabr.m}}
\end{minipage}}
\end{center}




\subsubsection[fc\_tools.graphics.DisplayFigures function]{\fcmcode{fc_tools.graphics.DisplayFigures} function}
The \fcmcode{fc_tools.graphics.DisplayFigures} function regularly distributes the figures on the screen.
\paragraph{Syntaxe}
\begin{syntaxe}
fc_tools.graphics.DisplayFigures()
fc_tools.graphics.DisplayFigures(n)
fc_tools.graphics.DisplayFigures('nfig',n)
\end{syntaxe}
%\paragraph{Description}
%\begin{description}
Without argument, all figures are regularly distributed on the screen.
Otherwise, empty figures with numbers \fcmcode{1} to \fcmcode{n} are created and regularly distributed on the screen.
%\end{description}


\subsubsection[fc\_tools.graphics.SaveAllFigsAsFiles function]{\fcmcode{fc_tools.graphics.SaveAllFigsAsFiles} function}
The \fcmcode{fc_tools.graphics.SaveAllFigsAsFiles} saves all figures as files.

\paragraph{Syntaxe}
\begin{syntaxe}
fc_tools.graphics.SaveAllFigsAsFiles(basename)
fc_tools.graphics.SaveAllFigsAsFiles(file, key, value, ...)
\end{syntaxe}


\paragraph{Description}
\begin{description}
\item \fbox{\fcmcode{fc_tools.graphics.SaveAllFigsAsFiles(basename)}} save each figure in the file 
$$\fcmcode{[basename,'_fig',fignumber]}$$
of the current directory where \fcmcode{fignumber} is the number of the figure to be saved.
\item \fbox{\fcmcode{fc_tools.graphics.SaveAllFigsAsFiles(file, key, value, ...)}} specifies function options using one or more 
\fcmcode{Name},\fcmcode{Value} pair 
arguments. Options are
\begin{itemize}
\item[$\bullet$] \texttt{'format'} : to specify the file format. \fcmcode{Value} could be \fcmcode{'epsc'} (default), \fcmcode{'pdf'},
\fcmcode{'png'} or \fcmcode{'pdflatex'}.
\item[$\bullet$] \texttt{'dir'} : to specify the directory (default \fcmcode{'.'}). the directory is created if it does not exist.
\item[$\bullet$] \texttt{'verbose'} :  if \fcmcode{true}, prints file names. Default is \fcmcode{false}.
\item[$\bullet$] \texttt{'tag'} : if \fcmcode{true} each figure is saved in the file:
$$\fcmcode{[basename,'_fig',fignumber,'_',software,version]}$$
where \fcmcode{software} is \fccmdname\ and  \fcmcode{version} is its release. Default is \fcmcode{false}.
\item[$\bullet$] \texttt{'size'} : to specify size of the image. Default is \fcmcode{[800,600]}.
\end{itemize}

\end{description}

\subsection{xcolor submodule}

\subsubsection[fc\_tools.graphics.xcolor.svg function]{\fcmcode{fc_tools.graphics.xcolor.svg} function}
The \fcmcode{fc_tools.graphics.xcolor.svg} function returns names and RGB values of the $149$ SVG colors.

\paragraph{Syntaxe}
\begin{syntaxe}
[name,rgb]=fc_tools.graphics.xcolor.svg()
\end{syntaxe}

\fcmcode{name} is cell array of string (color names) and \fcmcode{rgb} is $149$-by-$3$ array (rgb values)
such that the color \fcmcode{name\{i\}} has \fcmcode{rgb(i,:)} for rgb values.

\subsubsection[fc\_tools.graphics.xcolor.X11 function]{\fcmcode{fc_tools.graphics.xcolor.X11} function}
The \fcmcode{fc_tools.graphics.xcolor.X11} function returns names and RGB values of the $317$ X11 colors.

\paragraph{Syntaxe}
\begin{syntaxe}
[name,rgb]=fc_tools.graphics.xcolor.X11()
\end{syntaxe}

\fcmcode{name} is cell array of string (color names) and \fcmcode{rgb} is $317$-by-$3$ array (rgb values)
such that the color \fcmcode{name\{i\}} has \fcmcode{rgb(i,:)} for rgb values.

\subsubsection[fc\_tools.graphics.xcolor.fullX11 function]{\fcmcode{fc_tools.graphics.xcolor.fullX11} function}
The \fcmcode{fc_tools.graphics.xcolor.fullX11} function returns names and RGB values of the $738$ X11 colors.

\paragraph{Syntaxe}
\begin{syntaxe}
[name,rgb]=fc_tools.graphics.xcolor.fullX11()
\end{syntaxe}

\fcmcode{name} is cell array of string (color names) and \fcmcode{rgb} is $738$-by-$3$ array (rgb values)
such that the color \fcmcode{name\{i\}} has \fcmcode{rgb(i,:)} for rgb values.

\subsubsection[fc\_tools.graphics.xcolor.display\_colors function]{\fcmcode{fc_tools.graphics.xcolor.display_colors} function}
The \fcmcode{fc_tools.graphics.xcolor.display_colors} function displays SVG colors, X11 colors and (full)X11 colors.

\paragraph{Syntaxe}
\begin{syntaxe}
fc_tools.graphics.xcolor.display_colors()
\end{syntaxe}

\begin{filecontents*}{main.m}
fc_tools.graphics.xcolor.display_colors()
\end{filecontents*}
\begin{filecontents*}{after.m}
fc_tools.graphics.SaveAllFigsAsFiles('display_colors',SaveOptions{:})
\end{filecontents*}
\fcrun{figures/display_colors}

%{\tiny \verbatiminput{results/buildmesh2d01.txt}}
%\fcmatlaboutput{results/siMesh_plot01_\fccmdname\fccmdversionabr.m}{results/siMesh01.txt}
\begin{center}
\fbox{\begin{minipage}{0.95\textwidth}
\fcdblfigure{figures/display_colors_fig1_\fccmdname\fccmdversionabr}{figures/display_colors_fig2_\fccmdname\fccmdversionabr}
%\fcdblfigure{figures/plot2D_fig3_\fccmdname\fccmdversionabr}{figures/plot2D_fig4_\fccmdname\fccmdversionabr}
\fctrifigure{figures/display_colors_fig3_\fccmdname\fccmdversionabr}{figures/display_colors_fig4_\fccmdname\fccmdversionabr}
{figures/display_colors_fig5_\fccmdname\fccmdversionabr}
\par\noparindent\FClistingresize{\lstinputlisting[caption={\fcmcode{fc_tools.graphics.xcolor.display_colors} function \label{Lst:display_colors}}]
{figures/display_colors_\fccmdname\fccmdversionabr.m}}
\end{minipage}}
\end{center}

\subsection{monitors submodule}

\subsubsection[fc\_tools.graphics.monitors.onGrid function]{\fcmcode{fc_tools.graphics.monitors.onGrid} function}
The \fcmcode{fc_tools.graphics.monitors.onGrid} displays figures on a virtual \fcmcode{m}-by-\fcmcode{n} grid 
(as \fcmcode{subplot} command with axes) positioned on a selected monitor. 
\paragraph{Syntaxe}
\begin{syntaxe}
fc_tools.graphics.monitors.onGrid(n,m)
fc_tools.graphics.monitors.onGrid(n,m, key,value, ...)
\end{syntaxe}

\paragraph{Description}
\begin{description}
\item \fbox{\fcmcode{fc_tools.graphics.monitors.onGrid(n,m)}}\\
A virtual \fcmcode{m}-by-\fcmcode{n} grid is created on the first monitor and the figures numbered from 1 to \fcmcode{m*n}
are moved or created (if it doesn't exist) respectively to the position given by an index (default is the figure number).
This index runs row-wise; all columns of the first row are numbered from left to right and so on with rows $2$ to $\fcmcode{m}.$ 
%
\item \fbox{\fcmcode{fc_tools.graphics.monitors.onGrid(n,m, key,value)}}
specifies function options using one or more \fcmcode{key},\fcmcode{value} pair 
arguments. Options are
\begin{itemize}
\item[$\bullet$] \texttt{'figures'} : specifies the numbers of the figures to be used. 
Default is \fcmcode{1:m*n}.
\item[$\bullet$] \texttt{'positions'} : specifies the index on the grid corresponding to the \texttt{'figures'} option.
Default is \fcmcode{1:m*n}.
\end{itemize}
\end{description}

\begin{filecontents*}{main.m}
fc_tools.graphics.monitors.onGrid(2,3)
\end{filecontents*}
\begin{filecontents*}{after.m}
fc_tools.graphics.SaveAllFigsAsFiles('selectColors', SaveOptions{:})
\end{filecontents*}
\fcrun{figures/selectColors}


\subsubsection[fc\_tools.graphics.monitors.show function]{\fcmcode{fc_tools.graphics.monitors.show} function}
The \fcmcode{fc_tools.graphics.monitors.show} displays monitors with their resolution, position and number on
a virtual screen.
\paragraph{Syntaxe}
\begin{syntaxe}
fc_tools.graphics.monitors.show()
fc_tools.graphics.monitors.show(n,m)
fc_tools.graphics.monitors.show(n,m, key,value, ...)
\end{syntaxe}

\begin{filecontents*}{main.m}
fc_tools.graphics.display_rgb(rand(40,3))
\end{filecontents*}
\begin{filecontents*}{after.m}
fc_tools.graphics.SaveAllFigsAsFiles('display_rgb', SaveOptions{:})
\end{filecontents*}
\fcrun{figures/display_rgb}

\begin{center}
\fbox{\begin{minipage}{0.95\textwidth}
\fcimagec{figures/display_rgb_fig1_\fccmdname\fccmdversionabr}{0.7\textwidth}
\par\noparindent\FClistingresize{\lstinputlisting[caption={\fcmcode{fc_tools.graphics.display_rgb} function \label{Lst:display_rgb}}]
{figures/display_rgb_\fccmdname\fccmdversionabr.m}}
\end{minipage}}
\end{center}


\subsection{gptoolbox submodule}
This submodule contains some files of the \textbf{gptoolbox} from \textit{Alec Jacobson}
(see \url{https://github.com/alecjacobson/gptoolbox})

\subsection{crop submodule}
This submodule contains the function \fcmcode{crop} from \textit{Andrew Bliss}.

\subsection{vfield3 submodule}
This submodule contains the function \fcmcode{vfield3} from \textit{M MA} (see 
\url{https://www.mathworks.com/matlabcentral/fileexchange/8653-vfield3})
