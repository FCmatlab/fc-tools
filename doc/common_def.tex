% DON'T CHANGE THIS FILE if not in fc-tools package
%   automaticaly updated from fc-tools package with update_package command
% 

\newcommand{\gmsh}{\texttt{gmsh}}
\newcommand{\METIS}{\texttt{Metis}}
\newcommand{\Chaco}{\texttt{Chaco}}
\newcommand{\mooMesh}{\texttt{mooMesh}}
\newcommand{\fctools}{\fcfname{fc-tools}}
\newcommand{\fcamat}{\fcfname{fc-amat}}
\newcommand{\fcbench}{\fcfname{fc-bench}}
\newcommand{\fcoogmsh}{\fcfname{fc-oogmsh}}
\newcommand{\fchypermesh}{\fcfname{fc-hypermesh}}
\newcommand{\fcsimesh}{\fcfname{fc-simesh}}
\newcommand{\fcpsimesh}{\fcfname{fc-psimesh}}
\newcommand{\fcsiplt}{\fcfname{fc-siplt}}
\newcommand{\fcgraphicsformesh}{\fcfname{fc-graphics4mesh}}
\newcommand{\fcvfempun}{\fcfname{fc-vfemp1}}
\newcommand{\fcmesh}{\fcfname{fc-mesh}}
\newcommand{\fcfem}{\fcfname{fc-fem}}

\mdtheorem[linecolor=mypresii,linewidth=2pt,roundcorner=8pt,frametitlebackgroundcolor=mypresii!30,backgroundcolor=mypresii!20]
          {remark}[theorem]{\vcenteredinclude{\resizebox{0.75\logowidth}{!}{\bcinfo}}~remark}

\def\BaCo{\lambda} % Barycentric coordinates on d-simplex
\def\BaCoref{\femref{\BaCo}} % Barycentric coordinates on reference d-simplex
\def\Gradients{Gradients}
\def\GradientsVec{GradientsVec}
\def\FoncBase{\varphi}
\def\BasisFunc{\FoncBase}


\def\SaveDir{./figures}
\immediate\write18{mkdir -p \SaveDir}


\def\simpdim{{\rm dim}}
\def\simpd{{\rm d}}

\def\BDT{\TT{\Gamma}}
\def\nq{{\mathop{\rm n_q}\nolimits}}
\def\nqs{{\mathop{\rm n^2_q}\nolimits}}
\def\nme{{\mathop{\rm n_{me}}\nolimits}}
\def\nbe{{\mathop{\rm n_{be}}\nolimits}}
\def\dd{{\rm d}}
\def\q{{\rm q}}                                   % Nodes
\def\qL{\Local{\q}}                               % Local nodes
\def\ql{\mathop{\rm ql}\nolimits}       % Node labels
\def\me{\mathop{\rm me}\nolimits}                 % Mesh Elements
\def\mel{\mathop{\rm mel}\nolimits}
\def\be{\mathop{\rm be}\nolimits}                 % Boundary Elements
\def\bel{\mathop{\rm bel}\nolimits}                 % Boundary Elements label
%\def\Taires{\mathop{\rm areas}\nolimits}
\def\Taires{\mathop{\rm vol}\nolimits}
\def\areas{{\Taires}}
\def\ndf{{\mathop{\rm n_{dof}}\nolimits}}
%\def\ndfe{{\mathop{\rm n_{dfe}}\nolimits}}
\def\ndfe{{d+1}}
\def\Ndfe{{(d+1)}}
\def\ndfes{{\mathop{\rm n^2_{dfe}}\nolimits}}
\def\ndfs{{\mathop{\rm n^2_{df}}\nolimits}}
\def\ndof{{\mathop{\rm n_{dof}}\nolimits}}

\def\labels{\mathop{\it labels}\nolimits}  % précedemment lab
\def\nlabels{\mathop{\rm n_{lab}}\nolimits}
\newcommand{\Ilabels}[1]{ \SetFont{I}_{\labels}^{#1}}

\def\BasisFuncBa{\vecb{\psi}}    
\def\BasisFuncBb{\vecb{\psi}}
\def\area{{\rm area}}
\def\volume{{\rm vol}}
\def\volumes{{\rm vols}}
\newcommand{\modulo}[2]{ {#1}\mod{#2} }

\def\al{\mathop{\rm \nu}\nolimits}
\def\il{{\rm \alpha}}
\def\jl{{\rm \beta}}
\def\kl{{\rm \gamma}}
\def\ll{{\rm \mu}}

\newcommand{\simplex}{K} % d-simplex simplex element : Quarteroni's notation
\def\ddmun{{\mathrm{d}{-1}}}
\def\dsimplex{\rm{d}-simplex\ }
\def\dsimplices{\rm{d}-simplices\ }
\def\dmunsimplex{($\ddmun$)-simplex\ }
\def\dmunsimplices{($\ddmun$)-simplices\ }

% OBJECTS
\def\ooMeshElmt{\fcfname{siMeshElt}\ }
\def\siMeshElt{\ooMeshElmt}
\def\ooMesh{\fcfname{siMesh}\ }
\def\siMesh{\ooMesh}
\newcommand{\amat}{\fcfname{amat}\ }

\def\sTh{\texttt{sTh}}
\def\nsTh{\texttt{nsTh}}
\def\sThsimp{\texttt{sThsimp}}
\def\Gammatext{\texttt{Gamma}}
\def\sThlab{\texttt{sThlab}}
\def\PhysElmts{\texttt{PhysElts}}
\def\InterElmts{\texttt{InterElts}}
\def\toGlobal{\texttt{toGlobal}}
\def\toParent{\texttt{toParent}}
\def\nqParents{\texttt{nqParents}}
\def\toParents{\texttt{toParents}}
\newcommand{\fcsf}[2]{#1.#2}

\def\DOM{\Omega}
\def\DOMT{\TT{\Omega}} % \DOM\times[0,T]
\def\Th{{\cal T}_h}
\newcommand{\xt}[1]{ {\TT{\vecb{#1}}} }
\def\DOMH{{\Omega_h}}
\def\BD{\Gamma}
\def\BDH{\Gamma_h}
\newcommand\BDHC[1]{\Gamma_{h,#1}}
\def\BDMesh{\mathcal{B}_h}

\newcommand{\fcarray}[1]{{\fcvec{\rm #1}}}

% \newcommand{\fcmatlaboutput}[2]{
% \fclstsetmatlab
% \begin{center}
% \fbox{\begin{minipage}{0.95\textwidth}
% \lstinputlisting[numbers=none,captionpos=t,title={Matlab commands with output}]{#1}
% \lstinputlisting[
%   language={},numbers=none,frame=,backgroundcolor=\color{white},identifierstyle=\color{black},
%   basicstyle=\color{black}\tiny\ttfamily,keywordstyle=\color{black},
%   columns=flexible,
%   breaklines=true
% ]{#2}
% \end{minipage}}
% \end{center}
% \fclstsetmatlab
% }

\renewcommand{\fclstsetmatlab}{
 \lstset{language=MATLAB,backgroundcolor=\color{mypresiii!30},prebreak={ ...},breaklines=true,breakatwhitespace=true,basicstyle=\small,captionpos=above}
}

\definecolor{DarkBlue}{rgb}{.11,.23,.60}
\mdfdefinestyle{commandline}%
{
  leftmargin=5pt, rightmargin=10pt,innerleftmargin=15pt,
  middlelinecolor=DarkBlue,
  middlelinewidth=2pt,
  frametitlerule=false,
  backgroundcolor=black!10!white,
  frametitle={Command Window},
  frametitlefont={\normalfont\sffamily\color{white}\hspace{-1em}},
  frametitlebackgroundcolor=DarkBlue,
  singleextra={\draw[black!20,line width=12pt] 
        ($(O)+(7pt,1pt)$) --
        ($(O|-P)+(7pt,-\mdfframetitleboxtotalheight)-(0,1pt)$);
        \node[inner sep=0pt,color=black]at ($(O)+(7pt,9pt)$)%
        {$\scriptstyle f\!x$}; },
  nobreak=true,
}

\lstnewenvironment{script}{%
  \lstset{language=Matlab,basicstyle=\tiny\ttfamily,breaklines=true,%
          aboveskip=0pt,belowskip=0pt}
}{}
\surroundwithmdframed[style=commandline]{script}

\definecolor{DarkGrey}{rgb}{.1,.1,.1}

\mdfdefinestyle{syntaxestyle}{leftmargin=1cm,linecolor=black}%,frametitle={Syntaxe}}
\lstnewenvironment{syntaxe}{%
  \lstset{language=Matlab,breaklines=true,frame=,numbers=none,%
          aboveskip=0pt,belowskip=0pt}
}{}
\surroundwithmdframed[style=syntaxestyle]{syntaxe}

\ifthenelse{\equal{\fccmdname}{Octave}}{
\def\ToolboxPackageName{package}
\def\ToolboxPackageNames{packages}
}{
\def\ToolboxPackageName{toolbox}
\def\ToolboxPackageNames{toolboxes}
}

\mdfdefinestyle{terminalcommand}%
{
  leftmargin=0pt, rightmargin=0pt,innerleftmargin=5pt,
  middlelinecolor=mypresii,
  middlelinewidth=2pt,
  frametitlerule=false,
  backgroundcolor=white,
  nobreak=true,
}

\lstnewenvironment{ShellVerbatim}[1][] {%
  \lstset{language=bash,breaklines=true,frame=,numbers=none,basicstyle=\scriptsize\ttfamily,%
          aboveskip=0pt,belowskip=0pt,backgroundcolor=\color{white},#1}
}{}
\surroundwithmdframed[style=terminalcommand]{ShellVerbatim}

\definecolor{DarkBlue}{rgb}{.11,.23,.60}
\mdfdefinestyle{matlablike}%
{
  leftmargin=0pt, rightmargin=0pt,innerleftmargin=5pt,
  middlelinecolor=mypresii,
  middlelinewidth=2pt,
  frametitlerule=false,
  backgroundcolor=mypresii!10!white,
  nobreak=true,
}
\mdfdefinestyle{matlaboutput}%
{
  leftmargin=0pt, rightmargin=0pt,innerleftmargin=5pt,
  middlelinecolor=mypresii,
  middlelinewidth=1pt,
  frametitlerule=false,
  backgroundcolor=white,
  nobreak=true,
}

\lstnewenvironment{MatlabVerbatim}[1][]{%
  \lstset{language=bash,breaklines=true,basicstyle=\scriptsize\ttfamily,%
          aboveskip=0pt,belowskip=0pt,#1}
}{}
\surroundwithmdframed[style=matlablike]{MatlabVerbatim}

\lstnewenvironment{MatlabOutput}[1][]{%
  \lstset{language=bash,breaklines=true,basicstyle=\scriptsize\ttfamily,%
          aboveskip=0pt,belowskip=0pt,#1}
}{}
\surroundwithmdframed[style=matlaboutput]{MatlabOutput}

\mdfdefinestyle{octavelike}%
{
  leftmargin=0pt, rightmargin=0pt,innerleftmargin=5pt,
  middlelinecolor=mypresii!30,
  middlelinewidth=2pt,
  frametitlerule=false,
  backgroundcolor=mypresii!10!white,
  nobreak=true,
}

\lstnewenvironment{OctaveVerbatim}[1][]{%
  \lstset{language=Matlab,basicstyle=\scriptsize\ttfamily,breaklines=true,%
          aboveskip=0pt,belowskip=0pt,#1}
}{}
\surroundwithmdframed[style=octavelike]{OctaveVerbatim}

\mdfdefinestyle{octaveoutput}%
{
  leftmargin=0pt, rightmargin=0pt,innerleftmargin=5pt,
  middlelinecolor=mypresii,
  middlelinewidth=1pt,
  frametitlerule=false,
  backgroundcolor=white,
  nobreak=true,
}
\lstnewenvironment{OctaveOutput}[1][]{%
  \lstset{language=bash,breaklines=true,basicstyle=\scriptsize\ttfamily,%
          aboveskip=0pt,belowskip=0pt,#1}
}{}
\surroundwithmdframed[style=octaveoutput]{OctaveOutput}


\newcommand{\FClistingresize}[1]{
\par\noparindent\vspace{-0.25cm}
\begin{center}\fcresize{0.95}{1.4}{#1}
\end{center}
}


% #1 : caption
% #2 : listing
% #3 : output
\newcommand{\fcmatlablstwithoutput}[3]{
  \fclstsetmatlab
  \begin{center}
    \fbox{
      \begin{minipage}{0.95\textwidth}
        \FClistingresize{
          \lstinputlisting[captionpos={t},caption={  : #1}]{#2}
          \lstinputlisting[title={Output},captionpos={t},frame = single,
            language={},numbers=none,backgroundcolor=\color{white},identifierstyle=\color{black},
            basicstyle=\color{black}\small\ttfamily,keywordstyle=\color{black},
            columns=flexible,
            breaklines=true
          ]{#3}
        }
      \end{minipage}
    }
    \fclstsetmatlab
  \end{center}
}

% Add for fc-vfemp1
%\newcommand{\Pk}[1]{\mathbb{#1}}
