\documentclass[a4paper,10pt]{article}
\usepackage{etex}
\reserveinserts{28}
\usepackage{amscd}
\usepackage{amsmath}
\usepackage{amsfonts}
\usepackage{amssymb}
\usepackage{t1enc}
\usepackage{array}
%\usepackage[latin1]{inputenc}
\usepackage[utf8]{inputenc}
\usepackage{scrextend}

\usepackage{pdfpages}

\usepackage{fcenv}
\usepackage{fcalgo}
\usepackage{fcmaths,fcinfo,fcutils,fcexrun}
%\input{fcgmsh}

\usepackage{ifthen}
\usepackage{fancyvrb}
\usepackage{hyperref}

\input{special/cmd.tex}
\immediate\write18{rsync -avu special/before.m .}


\ifthenelse{\equal{\fccmdname}{Octave}}{
  \usepackage[theme=blue]{cosmetic}
  \colorlet{codecolor}{mypresii}
}{
  \usepackage[theme=orange]{cosmetic}
  \colorlet{codecolor}{mypresi}
}

%\def\FCTOOLBOXDIR{\fctoolboxdir}
% DON'T CHANGE THIS FILE if not in fc-tools package
%   automaticaly updated from fc-tools package with update_package command
% 

\newcommand{\gmsh}{\texttt{gmsh}}
\newcommand{\METIS}{\texttt{Metis}}
\newcommand{\Chaco}{\texttt{Chaco}}
\newcommand{\mooMesh}{\texttt{mooMesh}}
\newcommand{\fctools}{\fcfname{fc-tools}}
\newcommand{\fcamat}{\fcfname{fc-amat}}
\newcommand{\fcbench}{\fcfname{fc-bench}}
\newcommand{\fcoogmsh}{\fcfname{fc-oogmsh}}
\newcommand{\fchypermesh}{\fcfname{fc-hypermesh}}
\newcommand{\fcsimesh}{\fcfname{fc-simesh}}
\newcommand{\fcpsimesh}{\fcfname{fc-psimesh}}
\newcommand{\fcsiplt}{\fcfname{fc-siplt}}
\newcommand{\fcgraphicsformesh}{\fcfname{fc-graphics4mesh}}
\newcommand{\fcvfempun}{\fcfname{fc-vfemp1}}
\newcommand{\fcmesh}{\fcfname{fc-mesh}}
\newcommand{\fcfem}{\fcfname{fc-fem}}

\mdtheorem[linecolor=mypresii,linewidth=2pt,roundcorner=8pt,frametitlebackgroundcolor=mypresii!30,backgroundcolor=mypresii!20]
          {remark}[theorem]{\vcenteredinclude{\resizebox{0.75\logowidth}{!}{\bcinfo}}~remark}

\def\BaCo{\lambda} % Barycentric coordinates on d-simplex
\def\BaCoref{\femref{\BaCo}} % Barycentric coordinates on reference d-simplex
\def\Gradients{Gradients}
\def\GradientsVec{GradientsVec}
\def\FoncBase{\varphi}
\def\BasisFunc{\FoncBase}


\def\SaveDir{./figures}
\immediate\write18{mkdir -p \SaveDir}


\def\simpdim{{\rm dim}}
\def\simpd{{\rm d}}

\def\BDT{\TT{\Gamma}}
\def\nq{{\mathop{\rm n_q}\nolimits}}
\def\nqs{{\mathop{\rm n^2_q}\nolimits}}
\def\nme{{\mathop{\rm n_{me}}\nolimits}}
\def\nbe{{\mathop{\rm n_{be}}\nolimits}}
\def\dd{{\rm d}}
\def\q{{\rm q}}                                   % Nodes
\def\qL{\Local{\q}}                               % Local nodes
\def\ql{\mathop{\rm ql}\nolimits}       % Node labels
\def\me{\mathop{\rm me}\nolimits}                 % Mesh Elements
\def\mel{\mathop{\rm mel}\nolimits}
\def\be{\mathop{\rm be}\nolimits}                 % Boundary Elements
\def\bel{\mathop{\rm bel}\nolimits}                 % Boundary Elements label
%\def\Taires{\mathop{\rm areas}\nolimits}
\def\Taires{\mathop{\rm vol}\nolimits}
\def\areas{{\Taires}}
\def\ndf{{\mathop{\rm n_{dof}}\nolimits}}
%\def\ndfe{{\mathop{\rm n_{dfe}}\nolimits}}
\def\ndfe{{d+1}}
\def\Ndfe{{(d+1)}}
\def\ndfes{{\mathop{\rm n^2_{dfe}}\nolimits}}
\def\ndfs{{\mathop{\rm n^2_{df}}\nolimits}}
\def\ndof{{\mathop{\rm n_{dof}}\nolimits}}

\def\labels{\mathop{\it labels}\nolimits}  % précedemment lab
\def\nlabels{\mathop{\rm n_{lab}}\nolimits}
\newcommand{\Ilabels}[1]{ \SetFont{I}_{\labels}^{#1}}

\def\BasisFuncBa{\vecb{\psi}}    
\def\BasisFuncBb{\vecb{\psi}}
\def\area{{\rm area}}
\def\volume{{\rm vol}}
\def\volumes{{\rm vols}}
\newcommand{\modulo}[2]{ {#1}\mod{#2} }

\def\al{\mathop{\rm \nu}\nolimits}
\def\il{{\rm \alpha}}
\def\jl{{\rm \beta}}
\def\kl{{\rm \gamma}}
\def\ll{{\rm \mu}}

\newcommand{\simplex}{K} % d-simplex simplex element : Quarteroni's notation
\def\ddmun{{\mathrm{d}{-1}}}
\def\dsimplex{\rm{d}-simplex\ }
\def\dsimplices{\rm{d}-simplices\ }
\def\dmunsimplex{($\ddmun$)-simplex\ }
\def\dmunsimplices{($\ddmun$)-simplices\ }

% OBJECTS
\def\ooMeshElmt{\fcfname{siMeshElt}\ }
\def\siMeshElt{\ooMeshElmt}
\def\ooMesh{\fcfname{siMesh}\ }
\def\siMesh{\ooMesh}
\newcommand{\amat}{\fcfname{amat}\ }

\def\sTh{\texttt{sTh}}
\def\nsTh{\texttt{nsTh}}
\def\sThsimp{\texttt{sThsimp}}
\def\Gammatext{\texttt{Gamma}}
\def\sThlab{\texttt{sThlab}}
\def\PhysElmts{\texttt{PhysElts}}
\def\InterElmts{\texttt{InterElts}}
\def\toGlobal{\texttt{toGlobal}}
\def\toParent{\texttt{toParent}}
\def\nqParents{\texttt{nqParents}}
\def\toParents{\texttt{toParents}}
\newcommand{\fcsf}[2]{#1.#2}

\def\DOM{\Omega}
\def\DOMT{\TT{\Omega}} % \DOM\times[0,T]
\def\Th{{\cal T}_h}
\newcommand{\xt}[1]{ {\TT{\vecb{#1}}} }
\def\DOMH{{\Omega_h}}
\def\BD{\Gamma}
\def\BDH{\Gamma_h}
\newcommand\BDHC[1]{\Gamma_{h,#1}}
\def\BDMesh{\mathcal{B}_h}

\newcommand{\fcarray}[1]{{\fcvec{\rm #1}}}

% \newcommand{\fcmatlaboutput}[2]{
% \fclstsetmatlab
% \begin{center}
% \fbox{\begin{minipage}{0.95\textwidth}
% \lstinputlisting[numbers=none,captionpos=t,title={Matlab commands with output}]{#1}
% \lstinputlisting[
%   language={},numbers=none,frame=,backgroundcolor=\color{white},identifierstyle=\color{black},
%   basicstyle=\color{black}\tiny\ttfamily,keywordstyle=\color{black},
%   columns=flexible,
%   breaklines=true
% ]{#2}
% \end{minipage}}
% \end{center}
% \fclstsetmatlab
% }

\renewcommand{\fclstsetmatlab}{
 \lstset{language=MATLAB,backgroundcolor=\color{mypresiii!30},prebreak={ ...},breaklines=true,breakatwhitespace=true,basicstyle=\small,captionpos=above}
}

\definecolor{DarkBlue}{rgb}{.11,.23,.60}
\mdfdefinestyle{commandline}%
{
  leftmargin=5pt, rightmargin=10pt,innerleftmargin=15pt,
  middlelinecolor=DarkBlue,
  middlelinewidth=2pt,
  frametitlerule=false,
  backgroundcolor=black!10!white,
  frametitle={Command Window},
  frametitlefont={\normalfont\sffamily\color{white}\hspace{-1em}},
  frametitlebackgroundcolor=DarkBlue,
  singleextra={\draw[black!20,line width=12pt] 
        ($(O)+(7pt,1pt)$) --
        ($(O|-P)+(7pt,-\mdfframetitleboxtotalheight)-(0,1pt)$);
        \node[inner sep=0pt,color=black]at ($(O)+(7pt,9pt)$)%
        {$\scriptstyle f\!x$}; },
  nobreak=true,
}

\lstnewenvironment{script}{%
  \lstset{language=Matlab,basicstyle=\tiny\ttfamily,breaklines=true,%
          aboveskip=0pt,belowskip=0pt}
}{}
\surroundwithmdframed[style=commandline]{script}

\definecolor{DarkGrey}{rgb}{.1,.1,.1}

\mdfdefinestyle{syntaxestyle}{leftmargin=1cm,linecolor=black}%,frametitle={Syntaxe}}
\lstnewenvironment{syntaxe}{%
  \lstset{language=Matlab,breaklines=true,frame=,numbers=none,%
          aboveskip=0pt,belowskip=0pt}
}{}
\surroundwithmdframed[style=syntaxestyle]{syntaxe}

\ifthenelse{\equal{\fccmdname}{Octave}}{
\def\ToolboxPackageName{package}
\def\ToolboxPackageNames{packages}
}{
\def\ToolboxPackageName{toolbox}
\def\ToolboxPackageNames{toolboxes}
}

\mdfdefinestyle{terminalcommand}%
{
  leftmargin=0pt, rightmargin=0pt,innerleftmargin=5pt,
  middlelinecolor=mypresii,
  middlelinewidth=2pt,
  frametitlerule=false,
  backgroundcolor=white,
  nobreak=true,
}

\lstnewenvironment{ShellVerbatim}[1][] {%
  \lstset{language=bash,breaklines=true,frame=,numbers=none,basicstyle=\scriptsize\ttfamily,%
          aboveskip=0pt,belowskip=0pt,backgroundcolor=\color{white},#1}
}{}
\surroundwithmdframed[style=terminalcommand]{ShellVerbatim}

\definecolor{DarkBlue}{rgb}{.11,.23,.60}
\mdfdefinestyle{matlablike}%
{
  leftmargin=0pt, rightmargin=0pt,innerleftmargin=5pt,
  middlelinecolor=mypresii,
  middlelinewidth=2pt,
  frametitlerule=false,
  backgroundcolor=mypresii!10!white,
  nobreak=true,
}
\mdfdefinestyle{matlaboutput}%
{
  leftmargin=0pt, rightmargin=0pt,innerleftmargin=5pt,
  middlelinecolor=mypresii,
  middlelinewidth=1pt,
  frametitlerule=false,
  backgroundcolor=white,
  nobreak=true,
}

\lstnewenvironment{MatlabVerbatim}[1][]{%
  \lstset{language=bash,breaklines=true,basicstyle=\scriptsize\ttfamily,%
          aboveskip=0pt,belowskip=0pt,#1}
}{}
\surroundwithmdframed[style=matlablike]{MatlabVerbatim}

\lstnewenvironment{MatlabOutput}[1][]{%
  \lstset{language=bash,breaklines=true,basicstyle=\scriptsize\ttfamily,%
          aboveskip=0pt,belowskip=0pt,#1}
}{}
\surroundwithmdframed[style=matlaboutput]{MatlabOutput}

\mdfdefinestyle{octavelike}%
{
  leftmargin=0pt, rightmargin=0pt,innerleftmargin=5pt,
  middlelinecolor=mypresii!30,
  middlelinewidth=2pt,
  frametitlerule=false,
  backgroundcolor=mypresii!10!white,
  nobreak=true,
}

\lstnewenvironment{OctaveVerbatim}[1][]{%
  \lstset{language=Matlab,basicstyle=\scriptsize\ttfamily,breaklines=true,%
          aboveskip=0pt,belowskip=0pt,#1}
}{}
\surroundwithmdframed[style=octavelike]{OctaveVerbatim}

\mdfdefinestyle{octaveoutput}%
{
  leftmargin=0pt, rightmargin=0pt,innerleftmargin=5pt,
  middlelinecolor=mypresii,
  middlelinewidth=1pt,
  frametitlerule=false,
  backgroundcolor=white,
  nobreak=true,
}
\lstnewenvironment{OctaveOutput}[1][]{%
  \lstset{language=bash,breaklines=true,basicstyle=\scriptsize\ttfamily,%
          aboveskip=0pt,belowskip=0pt,#1}
}{}
\surroundwithmdframed[style=octaveoutput]{OctaveOutput}


\newcommand{\FClistingresize}[1]{
\par\noparindent\vspace{-0.25cm}
\begin{center}\fcresize{0.95}{1.4}{#1}
\end{center}
}


% #1 : caption
% #2 : listing
% #3 : output
\newcommand{\fcmatlablstwithoutput}[3]{
  \fclstsetmatlab
  \begin{center}
    \fbox{
      \begin{minipage}{0.95\textwidth}
        \FClistingresize{
          \lstinputlisting[captionpos={t},caption={  : #1}]{#2}
          \lstinputlisting[title={Output},captionpos={t},frame = single,
            language={},numbers=none,backgroundcolor=\color{white},identifierstyle=\color{black},
            basicstyle=\color{black}\small\ttfamily,keywordstyle=\color{black},
            columns=flexible,
            breaklines=true
          ]{#3}
        }
      \end{minipage}
    }
    \fclstsetmatlab
  \end{center}
}

% Add for fc-vfemp1
%\newcommand{\Pk}[1]{\mathbb{#1}}



%\def\ToolboxName{\fcbench}

\def\ToolboxName{\text{\includegraphics[height=1em]{\fclogoname}\ }}


\def\N{\texttt{N}}   % number of matrices
\def\nr{\texttt{nr}} % number of rows
\def\nc{\texttt{nc}} % number of column
\def\values{\texttt{values}} % array of N-by-nr-by-nc values


% \newcommand{\fccodedesc}[1]{\colorbox{mypresiii!15}{\lstinline[breaklines=false]{#1}}}
% \newcommand{\fccodedescbox}[1]{\mbox{\lstinline[breaklines=false]{#1}}}
% \newcommand{\fccodedesccolor}[2]{\colorbox{#1}{\color{mypresi}\lstinline[breaklines=false]{#2}}}
% \newcommand{\fccode}[1]{\text{\fccodedesccolor{white}{#1}}}
% \newcommand{\fcmcode}[1]{\mbox{{\color{mypresi}\lstinline[breaklines=false]{#1}}}}
% 
% \renewcommand{\fclstsetmatlab}{
% \lstset{language=MATLAB,%backgroundcolor=\color{mypresii!5},
%         frame=bt,
%         prebreak={ ...},breaklines=true,%
%         breakatwhitespace=true,
%         basicstyle=\color{mypresi}\ttfamily,%\small,
%         captionpos=above,commentstyle={\footnotesize\rmfamily\color{black}\rm},
%         escapeinside={}}
% }
\newcommand{\fccodedesc}[1]{\colorbox{mypresiii!15}{\lstinline[breaklines=false]{#1}}}
\newcommand{\fccodedescbox}[1]{\mbox{\lstinline[breaklines=false]{#1}}}
\newcommand{\fccodedesccolor}[2]{\colorbox{#1}{\color{codecolor}\lstinline[breaklines=false]{#2}}}
\newcommand{\fccode}[1]{\text{\fccodedesccolor{white}{#1}}}
\newcommand{\fcmcode}[1]{\mbox{{\color{codecolor}\lstinline[breaklines=false]{#1}}}}

\renewcommand{\fclstsetmatlab}{
\lstset{language=MATLAB,%backgroundcolor=\color{mypresii!5},
        frame=bt,
        prebreak={ ...},breaklines=true,%
        breakatwhitespace=true,
        basicstyle=\color{codecolor}\ttfamily,%\small,
        captionpos=above,commentstyle={\footnotesize\rmfamily\color{black}\rm},
        escapeinside={}}
}

% \IfFileExists{results/versions_\fccmdname\fccmdversionabr.out}{
%   \input{results/versions_\fccmdname\fccmdversionabr.out}
% }{
%   \newcommand{\fctoolsversion}{'dev'}
%   \newcommand{\fcbenchversion}{'dev'}
% }

\title{
\includegraphics[scale=0.5]{\fclogo}\\ \vspace{2.cm}
\ToolboxName \fccmdname\ \ToolboxPackageName, User's Guide\footnote{\fctitlefootnote.
}\\
\small{version \fctoolboxtag}
}

\author{François Cuvelier\thanks{LAGA,  UMR 7539, CNRS, Université Paris 13 - Sorbonne Paris Cité, Université Paris 8,
       99 Avenue J-B Clément, F-93430 Villetaneuse, France, cuvelier@math.univ-paris13.fr
       \newline \indent \indent This work was partially supported by the ANR project DEDALES under grant ANR-14-CE23-0005.}
} 
% 
% \title{
% \includegraphics[scale=0.5]{\fclogo}\\ \vspace{2.cm}
% \ToolboxName
% \fccmdname\ \ToolboxPackageName, User's Guide
% \footnote{Compiled with \fccmdname~\fccmdversion, with \ToolboxPackageNames\ \texttt{fc-bench}[\fcbenchversion]\ and \texttt{fc-tools}[\fctoolsversion]}
% }
% \author{François Cuvelier\thanks{Universit\'e Paris 13, Sorbonne Paris Cité, LAGA, CNRS UMR 7539,
%        99 Avenue J-B Clément, F-93430 Villetaneuse, France, cuvelier@math.univ-paris13.fr %} 
% %\and Gilles Scarella\hspace{0.15mm} \hspace{0.15mm}\thanks{Université Côte d'Azur, CNRS, LJAD, F-06108 Nice, France, 
% %       gilles.scarella@unice.fr. 
%       \newline \indent \indent This work was partially supported by the ANR project DEDALES under grant ANR-14-CE23-0005.} 
% } 
%\fvset{frame=lines,numbers=left,numbersep=3pt,fontshape=sl,fontsize}

\newcommand{\fcarrayset}[2]{
  \fcbmatsetdr{#2}{\K}{#1}
}
\newcommand{\fcarraysetspace}[3]{
  \fcbmatsetdr{#2}{#3}{#1}
}

\newcommand{\FCresize}[1]{
\par\noparindent
\begin{center}\fcresize{0.95}{1.4}{#1}
\end{center}
}


% \newcommand{\DDDarray}{3D-array}
% \newcommand{\DDDarrays}{3D-arrays}{
\renewcommand{\fcfnamefont}[1]{\ifmmode \text{\rmfamily\scshape#1}\else \rmfamily\scshape#1 \fi}


\newenvironment{fclstresize}{
  \begin{adjustbox}{width=0.8\textwidth,keepaspectratio}
  \begin{minipage}{1.2\textwidth}
  \begin{lstlisting}
}
{
  \end{lstlisting}
  \end{minipage}
  \end{adjustbox}
}

\immediate\write18{mkdir -p codes benchs results figures tabular special}

\setcounter{tocdepth}{2} % part,chapters,sections, subsections

\def\configremote{ssh://lagagit/MCS/Cuvelier/Matlab/fc-config}

\begin{document}

%\immediate\write18{ printf '%s' $(git ls-remote \configremote -g HEAD | cut -f 1)  }
%\immediate\write18{ echo '\string\\\string\\def\string\\\string\{'$(git ls-remote \configremote -g HEAD | cut -f 1)'\string\\\string\}' > fc-config.tex}
% \immediate\write18{ echo "\string\\\string\\def\string\\\string\\fcconfigcommit\string{$(git ls-remote ssh://lagagit/MCS/Cuvelier/Matlab/fc-config -g HEAD | cut -f 1)\string}" > fc-config.tex} 
% \immediate\write18{ echo "\string\\\string\\def\string\\\string\\fcconfigremote\string{ssh://lagagit/MCS/Cuvelier/Matlab/fc-config\string}" >> fc-config.tex} 



%%%%%%%%%%%%%%%
% To initialize title and version of packages
%   File before.m is create by "make before.m" or "make" command if in development mode
%   Otherwise it is create when using build_odoc or build_mdoc in fc-config package
\begin{filecontents*}{after.m}
\end{filecontents*}
\begin{filecontents*}{main.m}
[pkg,pkgs]=fc_tools.packages();
BuildVersions(pkg,pkgs);
\end{filecontents*}
\fcrun{codes/versions}[results/versions]
\input{results/versions_\fccmdname\fccmdversionabr.out}
%%%%%%%%%%%%%%%


\maketitle
\hspace{2.cm}

\begin{abstract}
The \ToolboxName \fccmdname\ \fctoolbox\ contains some basic tools used in my other \fctoolboxes.
\end{abstract}

\newpage
\tableofcontents

\newpage

% %>>>%%%% to initialize Matlab or Octave
% \IfFileExists{before.m}{}{
% \begin{filecontents*}{before.m}
% addpath ..
% fc_bench.init('verbose',0)
% \end{filecontents*}
% }
% %<<<%%%%
% \IfFileExists{after.m}{}{
% \begin{filecontents*}{after.m}
% \end{filecontents*}
% }

% \begin{filecontents*}{main.m}
% disp(sprintf('\\newcommand{\\fctoolsversion}{%s}',fc_tools.version()))
% disp(sprintf('\\newcommand{\\fcbenchversion}{%s}',fc_bench.version()))
% \end{filecontents*}
% \fcrun*{codes/versions}[results/versions]


\begin{filecontents*}{main.m}
fid=fopen('special/software.py','w');
fprintf(fid,'software=''%s''\n',fc_tools.sys.getSoftware());
fprintf(fid,'release=''%s''\n',fc_tools.sys.getRelease());
fclose(fid);
\end{filecontents*}
\fcrun{codes/software}

\fclstsetmatlab

\section{sys module}


\begin{itemize}
\item[$\bullet$] \fccode{fc_tools.sys.getComputerName()} returns the name of the computer as a string.
\item[$\bullet$] \fccode{fc_tools.sys.getUserName()} returns the name (login) of the current user as a string.
\item[$\bullet$] \fccode{fc_tools.sys.getRAM()} returns available memory (RAM) in GB of the computer.
\item[$\bullet$] \fccode{fc_tools.sys.getCPUinfo()} returns CPU(s) informations as a structure.
\item[$\bullet$] \fccode{fc_tools.sys.getOSinfo()} returns OS informations as a structure.
\end{itemize}

In Listing~\ref{lst:sys:01}, some examples are provided.
\begin{filecontents*}{main.m}
fprintf('RAM : %.2f GB\n',fc_tools.sys.getRAM())
CPU=fc_tools.sys.getCPUinfo()
OS=fc_tools.sys.getOSinfo()
\end{filecontents*}
\fcrun{codes/sys01}[results/sys01]
\fcmatlablstwithoutput{example using \fccode{fc_amat.sys} functions\label{lst:sys:01}}%
                      {codes/sys01_\fccmdname\fccmdversionabr.m}{results/sys01_\fccmdname\fccmdversionabr.out}


\section{graphics module}



\end{document}

\section{Introduction}
\input{latex/introduction.tex}

\section{Installation}
\input{latex/TestedOn_\fctoolboxtag_\fccmdname.tex}
\input{latex/installation_\fctoolboxtag_\fccmdname.tex}

\section[fc\_bench.bench function]{\fccode{fc_bench.bench} function}\label{sec:fc_bench.bench}
\input{latex/functions/bench.tex}


\subsection{Matricial product examples}
\input{latex/matricial_product.tex}

\subsection{LU factorization examples}
\input{latex/LU.tex}

\bibliography{biblio}
\bibliographystyle{plain}

\newpage

\begin{center}
{\Large\textbf{Informations for git maintainers of the \ToolboxName \fccmdname\ \ToolboxPackageName }}
\end{center}
\begin{filecontents*}{main.m}
Sep=[repmat('-',1,50),'\n'];
fprintf(Sep)
fc_tools.git.print_gitinfo(fc_bench.gitinfo())
fprintf(Sep)
fc_tools.git.print_gitinfo(fc_tools.gitinfo())
fprintf(Sep)
\end{filecontents*}
\fcrun{codes/gitinfo}[results/gitinfo]

\begin{center}
    \fbox{
      \begin{minipage}{0.95\textwidth}
        \FClistingresize{
          \lstinputlisting[title={git informations on the \ToolboxPackageNames\ used to build this manual},captionpos={t},frame = single,
            language={},numbers=none,backgroundcolor=\color{white},identifierstyle=\color{black},
            basicstyle=\color{black}\small\ttfamily,keywordstyle=\color{black},
            columns=flexible,
            breaklines=true
          ]{results/gitinfo_\fccmdname\fccmdversionabr.out}
        }
      \end{minipage}
    }
    \fclstsetmatlab
  \end{center}

\begin{filecontents*}{main.m}
Sep=[repmat('-',1,50),'\n'];
fprintf(Sep)
fc_tools.git.print_gitinfo(fc_tools.git.get_info('/home/cuvelier/texmf/tex/latex/fctools/'))
fprintf(Sep)
\end{filecontents*}
\fcrun{codes/gitinfo_latex}[results/gitinfo_latex]

\begin{center}
    \fbox{
      \begin{minipage}{0.95\textwidth}
        \FClistingresize{
          \lstinputlisting[title={git informations on the \LaTeX\ package used to build this manual},captionpos={t},frame = single,
            language={},numbers=none,backgroundcolor=\color{white},identifierstyle=\color{black},
            basicstyle=\color{black}\small\ttfamily,keywordstyle=\color{black},
            columns=flexible,
            breaklines=true
          ]{results/gitinfo_latex_\fccmdname\fccmdversionabr.out}
        }
      \end{minipage}
    }
    \fclstsetmatlab
  \end{center}

% \IfFileExists{./fc-config.tex}{}{
%   \immediate\write18{ echo "\string\\\string\\def\string\\\string\\fcconfigcommit\string{$(git ls-remote ssh://lagagit/MCS/Cuvelier/Matlab/fc-config -g HEAD | cut -f 1)\string}" > fc-config.tex} 
%   \immediate\write18{ echo "\string\\\string\\def\string\\\string\\fcconfigremote\string{ssh://lagagit/MCS/Cuvelier/Matlab/fc-config\string}" >> fc-config.tex} 
% }
% \input{fc-config.tex}

\IfFileExists{./special/fc-config.tex}{}{
  \immediate\write18{ echo "\string\\\string\\def\string\\\string\\fcconfigcommit\string{$(git ls-remote ssh://lagagit/MCS/Cuvelier/Matlab/fc-config -g HEAD | cut -f 1)\string}" > special/fc-config.tex} 
  \immediate\write18{ echo "\string\\\string\\def\string\\\string\\fcconfigremote\string{ssh://lagagit/MCS/Cuvelier/Matlab/fc-config\string}" >> special/fc-config.tex} 
}
\input{special/fc-config.tex}


Using the remote configuration repository:
\begin{center}
    \fbox{
      %\begin{minipage}{0.95\textwidth}
        \fcresize{0.93}{1.3}{
\begin{tabular}{ll}
 url & \texttt{\fcconfigremote}\\
 commit& \texttt{\fcconfigcommit}
\end{tabular}
}
    }
    \fclstsetmatlab
  \end{center}
  
 \end{document}
