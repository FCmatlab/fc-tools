\immediate\write18{mkdir -p \OHTDIR}

\begin{presentation}
\newline
The \textbf{fc-tools} Octave package contains some functions used in various packages I develop: \textbf{fc-oogmsh}, \textbf{fc-oomesh}, ...
\newline

\end{presentation}

\subsubsection{Requirements}
This package needs \textbf{Octave} version greater or equal to \textbf{4.0.0}.

\subsubsection{Current development release}

\immediate\write18{mkdir -p \OHTDIR/0.0.12}
\begin{tabular}{|M|M|M|M|M|}
\hline \\ 
\textbf{Version} & \textbf{date} & \textbf{package} & \textbf{archives} & \textbf{pdf} \\ \hline
0.0.12 & December 14, 2016 &
\BuildLinkWithSizeInKo{\IHTDIR/distrib/0.0.12/fc-tools-0.0.12.tar.gz}
                  {\OHTDIR/fc-tools-0.0.12.tar.gz}
                  {\includegraphics*{images/icons/tar-gz_40.png}}
& 
\begin{tabular}{l}
\BuildLinkWithSizeInKo{\IHTDIR/distrib/0.0.12/ofc-tools-0.0.12.tar.gz}
                  {\OHTDIR/ofc-tools-0.0.12.tar.gz}
                  {\includegraphics*{images/icons/tar-gz_40.png}}
\\ 
\BuildLinkWithSizeInKo{\IHTDIR/distrib/0.0.12/ofc-tools-0.0.12.zip}
                  {\OHTDIR/ofc-tools-0.0.12.zip}
                  {\includegraphics*{images/icons/zip_new40.png}} 
\\ 
\BuildLinkWithSizeInKo{\IHTDIR/distrib/0.0.12/ofc-tools-0.0.12.7z}
                  {\OHTDIR/ofc-tools-0.0.12.7z}
                  {\includegraphics*{images/icons/7z-icon-40.png}}                   
\end{tabular}
&
to do!
\\ \hline
\end{tabular}

This package was tested under Ubuntu 14.04 LTS with Octave versions 4.0.0 to 4.0.3 and 4.2.0. The installation of Octave is described 
\href{http://www.math.univ-paris13.fr/~cuvelier/Octave.html}{here}.

\subsubsection{Installation of the fc-tools package}

\begin{itemize}
\item Download the packages. For example, in a terminal:
\begin{verbatim}
$ wget http://www.math.univ-paris13.fr/~cuvelier/software/codes/fc-tools/0.0.12/fc-tools-0.0.12.tar.gz
\end{verbatim}
\item Under Octave :
\begin{verbatim}
>> pkg install fc-tools-0.0.12.tar.gz
\end{verbatim}
\item Now to use \texttt{fc-tools} in any Octave session, it is necessary to load the package:
\begin{verbatim}
>> pkg load fc-tools
\end{verbatim}
\end{itemize}

\subsubsection{Uninstalling the fc-tools package}
In an Octave session, unistalling the \texttt{fc-tools} package is done by:
\begin{verbatim}
>> pkg uninstall fc-tools
\end{verbatim}

\subsubsection{Installation from archives}
The archives are proposed in three formats zip, 7z and tar.gz. These archives are not Octave packages and cannot be installed with \texttt{pkg} command.
The installation process follows:
\begin{itemize}
\item Download one of the archives
\item Uncompress the archive
\item Add path to Octave
\end{itemize}

For example, with the tar.gz file, downloading and uncompressing are done by 
\begin{verbatim}
$ wget http://www.math.univ-paris13.fr/~cuvelier/software/codes/fc-tools/0.0.12/ofc-tools-0.0.12.tar.gz
$ tar zxvf ofc-tools-0.0.12.tar.gz
ofc-tools-0.0.12/
ofc-tools-0.0.12/+fcTools/
ofc-tools-0.0.12/+fcTools/init.m
ofc-tools-0.0.12/+fcTools/+graphics/
ofc-tools-0.0.12/+fcTools/+graphics/SaveAllFigsAsFiles.m
ofc-tools-0.0.12/+fcTools/+graphics/SetHoldOn.m
ofc-tools-0.0.12/+fcTools/+graphics/RestoreHold.m
ofc-tools-0.0.12/+fcTools/+graphics/selectColors.m
ofc-tools-0.0.12/+fcTools/+graphics/colorstr2rgb.m
ofc-tools-0.0.12/+fcTools/+utils/
ofc-tools-0.0.12/+fcTools/+utils/deleteCellOptions.m
ofc-tools-0.0.12/+fcTools/+sys/
ofc-tools-0.0.12/+fcTools/+sys/isfileexists.m
ofc-tools-0.0.12/+fcTools/+comp/
ofc-tools-0.0.12/+fcTools/+comp/AddRequired.m
ofc-tools-0.0.12/+fcTools/+comp/isOldParser.m
ofc-tools-0.0.12/+fcTools/+comp/Parse.m
ofc-tools-0.0.12/+fcTools/+comp/isOctave.m
ofc-tools-0.0.12/+fcTools/+comp/AddParamValue.m
ofc-tools-0.0.12/+fcTools/version.m
ofc-tools-0.0.12/DESCRIPTION
ofc-tools-0.0.12/COPYING
$
\end{verbatim}
Then under Octave, one can use \texttt{addpath} command with the \texttt{ofc-tools-0.0.12} full path:
\begin{verbatim}
>> addpath('<FULL PATH>/ofc-tools-0.0.12')
\end{verbatim}
 






